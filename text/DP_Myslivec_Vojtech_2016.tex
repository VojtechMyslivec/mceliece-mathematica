% options:
% thesis=B bachelor's thesis
% thesis=M master's thesis
% czech thesis in Czech language
% slovak thesis in Slovak language
% english thesis in English language
% hidelinks remove colour boxes around hyperlinks

\documentclass[thesis=M,czech,hidelinks]{FITthesis}[2012/06/26]

\usepackage[utf8]{inputenc} % LaTeX source encoded as UTF-8

\usepackage{graphicx} %graphics files inclusion
% \usepackage{amsmath} %advanced maths
% \usepackage{amssymb} %additional math symbols

\usepackage{dirtree} %directory tree visualisation
% \usepackage{textcomp} % \textcopyleft

% % list of acronyms
% \usepackage[acronym,nonumberlist,toc,numberedsection=autolabel]{glossaries}
% \iflanguage{czech}{\renewcommand*{\acronymname}{Seznam pou{\v z}it{\' y}ch zkratek}}{}
% \makeglossaries

\newcommand{\tg}{\mathop{\mathrm{tg}}} %cesky tangens
\newcommand{\cotg}{\mathop{\mathrm{cotg}}} %cesky cotangens

% % % % % % % % % % % % % % % % % % % % % % % % % % % % % %
% ODTUD DAL VSE ZMENTE
% % % % % % % % % % % % % % % % % % % % % % % % % % % % % %
\department{Katedra počítačových systémů}
\title{Asymetrický šifrovací algoritmus McEliece}
\authorGN{Vojtěch} %(křestní) jméno (jména) autora
\authorFN{Myslivec} %příjmení autora
\authorWithDegrees{Bc. Vojtěch Myslivec} %jméno autora včetně současných akademických titulů
\supervisor{prof. Ing. Róbert Lórencz, CSc.}
%TODO
%\acknowledgements{Rodina, Lórencz, Kalvoda, Guth}
%TODO
\abstractCS{V~několika větách shrňte obsah a přínos této práce v~češtině. Po přečtení abstraktu by se čtenář měl mít čtenář dost informací pro rozhodnutí, zda chce Vaši práci číst.}
%TODO
\abstractEN{Sem doplňte ekvivalent abstraktu Vaší práce v~angličtině.}
\placeForDeclarationOfAuthenticity{V~Praze}
\declarationOfAuthenticityOption{4} %volba Prohlášení (číslo 1-6)
\keywordsCS{McEliece, asymetrická kryptografie, postkvantová kryptografie,
binární Goppa kódy, konečná tělesa}
\keywordsEN{McEliece, public-key cryptography, post-quantum cryptography, binary
Goppa codes, finite fields}

\begin{document}

% \newacronym{CVUT}{{\v C}VUT}{{\v C}esk{\' e} vysok{\' e} u{\v c}en{\' i} technick{\' e} v Praze}
% \newacronym{FIT}{FIT}{Fakulta informa{\v c}n{\' i}ch technologi{\' i}}

\begin{introduction}
        %sem napište úvod Vaší práce
\end{introduction}

\chapter{Cíl práce}

\chapter{Konečná tělesa}


\chapter{Lineární kódy}

\chapter{Kryptosystém McEliece}

\chapter{Implementace}

\begin{conclusion}
        %sem napište závěr Vaší práce
\end{conclusion}

\bibliographystyle{csn690}
\bibliography{mybibliographyfile}

\appendix

\chapter{Seznam použitých zkratek}
% \printglossaries
\begin{description}
        \item[...]
        %\item[TLS] Three Letter Shortcut
\end{description}

\chapter{Obsah přiloženého CD}

%upravte podle skutecnosti

%TODO
\begin{figure}
        \dirtree{%
                .1 readme.txt\DTcomment{stručný popis obsahu CD}.
                .1 exe\DTcomment{adresář se spustitelnou formou implementace}.
                .1 src.
                .2 impl\DTcomment{zdrojové kódy implementace}.
                .2 thesis\DTcomment{zdrojová forma práce ve formátu \LaTeX{}}.
                .1 text\DTcomment{text práce}.
                .2 thesis.pdf\DTcomment{text práce ve formátu PDF}.
                .2 thesis.ps\DTcomment{text práce ve formátu PS}.
        }
\end{figure}

\end{document}
